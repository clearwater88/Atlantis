%%%%%%%%%%%%%%%%%%%%%%%%%%%%%%%%%%%%%%%%%
% Short Sectioned Assignment
% LaTeX Template
% Version 1.0 (5/5/12)
%
% This template has been downloaded from:
% http://www.LaTeXTemplates.com
%
% Original author:
% Frits Wenneker (http://www.howtotex.com)
%
% License:
% CC BY-NC-SA 3.0 (http://creativecommons.org/licenses/by-nc-sa/3.0/)
%
%%%%%%%%%%%%%%%%%%%%%%%%%%%%%%%%%%%%%%%%%

%----------------------------------------------------------------------------------------
%	PACKAGES AND OTHER DOCUMENT CONFIGURATIONS
%----------------------------------------------------------------------------------------

\documentclass[paper=a4, fontsize=11pt]{article} % A4 paper and 11pt font size

\usepackage[T1]{fontenc} % Use 8-bit encoding that has 256 glyphs
\usepackage[english]{babel} % English language/hyphenation
\usepackage{amsmath,amsfonts,amsthm} % Math packages
\usepackage{subfigure}
\usepackage{placeins}
\usepackage{longtable}

\usepackage[pdftex]{graphicx}
\usepackage{epstopdf}

\usepackage{times}
\usepackage{graphicx}
\setlength{\headheight}{13.6pt} % Customize the height of the header

%\usepackage{float}
%\floatstyle{boxed} 
%\restylefloat{figure}

\numberwithin{equation}{section} % Number equations within sections (i.e. 1.1, 1.2, 2.1, 2.2 instead of 1, 2, 3, 4)
\numberwithin{table}{section} % Number tables within sections (i.e. 1.1, 1.2, 2.1, 2.2 instead of 1, 2, 3, 4)

\setlength\parindent{0pt} % Removes all indentation from paragraphs - comment this line for an assignment with lots of text

%----------------------------------------------------------------------------------------
%	TITLE SECTION
%----------------------------------------------------------------------------------------

\newcommand{\horrule}[1]{\rule{\linewidth}{#1}} % Create horizontal rule command with 1 argument of height

\title{	
\normalfont \normalsize 
\horrule{0.5pt} \\[0.4cm] % Thin top horizontal rule
\huge Line model \\ % The assignment title
\horrule{2pt} \\[0.5cm] % Thick bottom horizontal rule
}

\author{Jeroen Chua} % Your name

\date{\normalsize\today} % Today's date or a custom date

\begin{document}

\maketitle % Print the title

\section{Notation}

\begin{itemize}
\item $\beta$: Finite set of bricks
\item $t$: brick index (we will imagine that associate with each brick in the active set is an index).
\item $s_t \in [0,1]$: random variable indicating on/off state of brick $t$
\item $s_{1:t} \in [0,1]^t$: set of random variables on/off state of bricks $[1..t]$.
\item $x_t$ : pose of brick $t$.
\item $G_{t,i} \in 0 \cup [1..G] $: random variable indicating whether brick $i$ is the \textbf{child} of brick $t$, and if so, which of the $G$ slots it is in. $0$ indicates not a child, any other value indicates slot number.
\item $R_{t} = \{G_{i,t}, \forall i\}$: set of random variables indicating whether each brick is a parent of $t$.

\end{itemize}

\section{Model}




%\begin{table}[h]
%\caption{Generated data with description of the data and subject answers. Although potentially useful, to save space, repeat answers were removed.} %title of the table
%\centering
%% centering table
%\begin{tabular}{|c|c|c|}
%\hline
%dataset & description & human answers \\
%\hline
%$D=[1,6,2,24]$ & Factorials & $[1,2,6,24]$ \\
%\hline
%$D=[31,41,59]$ & Digits of $\pi$, broken into $2$-digit numbers & $[17,10,23]$ \\
%\hline
%$D=[1,4,9,16]$ & Perfect squares & $[25,36,49,64]$ \\
%\hline
%\end{tabular}
%\label{data}
%\end{table}



%\begin{figure}[htbp]
%\begin{center}
%\includegraphics[width=\textwidth]{Q1eigenvalue.eps}
%\caption{Eigenvalues of training faces dataset. Values sorted in decreasing order.}
%\label{Q1eigenval}
%\end{center}
%\end{figure}

%\FloatBarrier
\end{document}