\documentclass[11pt]{article}
\usepackage{fullpage}
\usepackage{amsmath}
\usepackage{amssymb}
\usepackage{graphicx}
\usepackage{color}
\usepackage{float}
\usepackage{caption}
\usepackage{cases}
\usepackage{placeins}

\renewcommand\baselinestretch{1.5}
\newcommand{\Pose}{{\cal P}}
\newcommand{\A}{{\cal A}} 
\newcommand{\X}{\mathbf{X}} 
\newcommand{\symb}{\Sigma}
\newcommand{\actP}{P^{\A}}
\newcommand{\B}{\cal B}
\newcommand{\Xrm}{\X^{\Am}_{-r}}
\newcommand{\XrmPrev}{\X^{\A_{m-1}}_{-r}}
\newcommand{\Am}{\A_{m}} 
\newcommand{\Real}{\mathbb{R}}
\newcommand{\gcomp}{comp(\mathbf{g}^{\Am})}
\newcommand{\gcompPrev}{comp(\mathbf{g}^{\A_{m-1}})}
\newcommand{\qb}{\vec{q}_b}
\newcommand{\qa}{\vec{q}_a}
%\renewcommand{\prod}{\prod\limits}

\newenvironment{definition}[1][Definition]{\begin{trivlist}
\item[\hskip \labelsep {\bfseries #1}]}{\end{trivlist}}

\begin{document}

\title{Jeroen's model}
\maketitle

\section{Notation}

\begin{itemize}

\item $\symb$: partially-ordered list of symbols (types).

%\item $\symb_i$: ith symbol in partial ordering.

\item $T$: number of symbols in grammar.

\item $\Pose$: pose space.

\item ${\B} = (b_1,\ldots,b_N)$: set of bricks.

\item $\Pose(b) \in \Pose$: set of poses associated with brick $b$.

\item $\A = (a_1,\ldots,a_M)$, $\A \subseteq {\B}$: set of active bricks.

\item $\Am = (a_1,\ldots,a_m)$, $m \leq M$ $\Am \subseteq {\A}$: subset of active bricks.

\item $t(b)$: type of brick $b$.

\item $\#(C)$: the cardinality (size) of some set $C$. \emph{E.g.} $\#(\A)=M$.

\item ${\cal R}$: set of production rules of the form
$A \rightarrow B_1,\ldots,B_k$ with $r = A,B_1,\ldots,B_k \in \Sigma$. ${\cal R}$ is acyclic.

\item $n(b)$: number of slots/children associated with bricks of type $t(b)$.

%\item ${\cal R}(b)$: rules with $t(b) \in \B$ in the LHS.
\item $c(b)$: the set of production rules with $t(b)$ in the LHS.

%\item $p_b$: distribution over ${\cal R}(b)$.
\item $\vec{\theta}_b \in \mathbb{R}^{\#(c(b))}$: distribution over rules with  $t(b) \in \B$ in the LHS.

%\item $p_{r,n}(x_i|x)$, $r \in {\cal R}$, $x,x_i \in {\cal P}$, $n \in [1...n(r)]$: conditional probability density over pose for $B_i$ given pose for $A$ for this rule and slot.

\item $s_b \in \{0,1\}$: on/off state of brick $b$.

\item $r_b \in {\cal R}(t(b))$: rule used by brick $b$.

\item $\mathbf{g_b} \in ( {\B} \cup \perp )^{n(b)}$: array of pointers, where each element can either point to a brick, or null (no child).

\item $\vec{q}_b \in \{\Pose(b), \perp\}$: pose for brick $b$. $\vec{q}_b = \perp$ is a ``nothing'' pose that does not contribute to image evidence.

\item $\mathbf{q}^{\Am}$: poses for bricks $a \in \Am$.

\item $\mathbf{s^{\Am}}$: on/off state for bricks $a \in \Am$.

\item $\mathbf{r^{\Am}}$: rules for bricks $a \in \Am$.

\item $\mathbf{g}_b^{\Am} \in ( {\Am} \cup \emptyset )^{n(b)}$: array of pointers, where each element can either point to a brick in the active set defined by $\Am$, or blank (child not yet specified) denoted by $\emptyset$. Note that blank is \textbf{different} than null (no child) in that blank means ``there could be a child in this slot, but we don't know if there is or which one yet'' while null (no child) means ``there definitely isn't a child in this slot''.

\item $\gcomp$: the elements of $\mathbf{g}^{\Am}$ for which $g_{b,k}^{\Am} = \emptyset$. Note: we use ``comp'' as the identifier for this concept since these are the slots that may receive a child in the future; these are the slots to be``completed''.

\item $H^{\X_{m-1}} (\vec{g}^{\Am}_b)$: \textbf{set} of children that $(\vec{g}^{\Am}_b)$ points to that were orphans according to $\X_{m-1}$.

\item $\tilde{H}^{\X_{m-1}} (\vec{g}^{\Am}_b)$: \textbf{multi-set} of children that $(\vec{g}^{\Am}_b)$ points to that were orphans according to $\X_{m-1}$.

\item $V(b)$: ``before'' set of bricks, which we define as $V(b) = \{b' : t(b') < t(b) \}$ where $<$ means ``comes before in the partial ordering defined by $\symb$''.

\item $W(b)$: ``after'' set of bricks, which we define as $V(b) = \{b' : t(b') > t(b) \}$ where $>$ means ``comes after in the partial ordering defined by $\symb$''.

\item $\vec{\phi}_{b,k,r} \in \mathbb{R}^{\#(W(b))+1}$: distribution over bricks to which brick $b$'s $k$th slot points to if brick $b$ is using rule $r$. Note that brick $b$ may only point to bricks in its ``after'' set. Note that the $+1$ is present since a brick/slot may point to nothing, denoted by the symbol $\perp$. \emph{edit}: this is a bit sloppy. Each slot has a particular type, and only bricks of that type may go in this slot. The set of possibilities for each slot, in general, is a subset of the ``after'' set, and not the entire after set.

\item $\vec{\pi} \in \Real^T$: vector of mixing coefficients for the templates of each type of brick.

\item $\X = \{ \mathbf{s}, \mathbf{r}, \mathbf{g}, \mathbf{q}\}$: collection of hidden variables of model.

\item $\X^{\Am} = \{ \mathbf{s^{\Am}}, \mathbf{r^{\Am}}, \mathbf{g^{\Am}}, \mathbf{q^{\Am}} \}$: collection of hidden variables for bricks $a \in \Am$.

\item $\Xrm = \{ \mathbf{s^{\Am}}, \mathbf{g^{\Am}}, \mathbf{q^{\Am}} \}$: collection of hidden variables for bricks $a \in \Am$ except the hidden variables representing rules used by the bricks.

\item $orph(\Xrm)$: the set of bricks that are orphans according to the assignments in $\Xrm$ (see definition of ``orphan'' below).

\item $off(\Xrm)$: the set of bricks that are off (have $s = 0$) according to the assignments in $\Xrm$. Note that if a brick is off, then $\nexists \{a,k\}$, $a \in \Am$ s.t. $g^{\Am}_{a,k} = b$ where $b$ is the off-brick in question.

\item $Y$: the image.

\end{itemize}

\begin{definition}
\textbf{Orphan}: A brick $b$ is an \textbf{orphan} according to the assignment of random variables for $\Xrm$ iff $s_b=1$ and $\nexists \{a,k\}$ s.t. $g^{\Am}_{a,k} = b$ according to $\Xrm$.
\end{definition}


\section{Model}

\begin{figure}[htbp]
\begin{center}
\includegraphics[width=\textwidth, trim=0cm 6cm 0cm 0cm]{gm.pdf}
\caption{Graphical model}
\label{gm}
\end{center}
\end{figure}

\FloatBarrier

The decomposition of the joint probability is given by the graphical model in Fig. \ref{gm} and by: \\

\begin{eqnarray}
P(\X, Y) &=& \big(\prod\limits_{b \in {\B}} P(s_b \mid \mathbf{g}_{V(b)}) \big ) \\
& & \times \big( \prod_{b \in {\B}} P(r_b \mid s_b) \prod_{k=1}^{n(b)} P(g_{b,k} \mid r_b) \big) \nonumber \\
& & \times \big( \prod_{b \in {\B}} P(\qb \mid s_b)\big) \nonumber \\
& & \times \big(P(Y \mid \mathbf{q}) \big) \nonumber 
\end{eqnarray}
%
where we define:
%
\begin{subnumcases}{\label{eqn.probS} P(s_b = 1 \mid \mathbf{g}_{V(b)})=} 
 1, & if $\exists$ \{a,k\} s.t. $g_{a,k} = b$ , $g_{a,k} \in {g}_{V(b)}$  \\
 \epsilon, & if $\nexists$ \{a,k\} s.t. $g_{a,k} = b$ , $g_{a,k} \in {g}_{V(b)}$
\end{subnumcases}
%
\begin{eqnarray}
P(r_b \mid s_b) &=& Discrete(\vec{\theta}_b) \\
P(g_{b,k} \mid r_b) &=& Discrete(\vec{\phi}_{b,k,r_b})
\end{eqnarray}
%
\begin{subnumcases}{P(\qb = \vec{q} \mid s_b)\big)}
\frac{1}{\#\Pose(b)}, & if $s_b = 1$, $\vec{q} \in \Pose(b)$ \\
0, & if $s_b = 1$ and $\vec{q} = \perp$ \\
0, & if $s_b = 0$, $\vec{q} \in \Pose(b)$ \\
1, & if $s_b = 0$ and $\vec{q} = \perp$
\end{subnumcases}
%
\begin{eqnarray}
P(Y \mid \mathbf{q}) &=& \prod_{p} P(Y_p \mid \mathbf{q}) \label{eqn.Yq} \\
P(Y_p \mid \mathbf{q}) &\propto& \sum_{b} \pi_{t(b)}P(Y_p \mid \qb) I(\text{template overlaps with pixel } p) \\
P(Y_p \mid \qb) &=& \text{defn here, with Bernoulli distribution and stuff}.
\end{eqnarray}

where $I(\cdot)$ is an indicator variable. 


\section*{Localization}

%For localization, we do not explicitly represent $\mathbf{r}$; instead we express the state configuration in terms of $\mathbf{g}_b^{\A}$ only, which implicitly specifies a set of compatible rules. Rather than selecting specific values for $\mathbf{r}$, we instead marginalize over the set of compatible rules defined by $\mathbf{g^{\A}}$ wherever  $\mathbf{r}$ is referred to.

We define localizations in terms of the components $\big( \prod_{b \in {\B}} P(s_b \mid \mathbf{g}_{V(b)}) \big )$, $\big( \prod_{b \in {\B}} P(r_b \mid s_b) \prod_{k=1}^{n(b)} P(g_{b,k} \mid r_b) \big)$, $\big( \prod_{b \in {\B}} P(\qb \mid s_b)\big)$, and $P(Y \mid \mathbf{q})$ separately.

\begin{eqnarray}
\big(\prod_{b \in {\B}} P(s_b \mid \mathbf{g}_{V(b)}) \big ) &\Rightarrow& \big(\prod_{a \in {\Am}} P(s_a \mid \mathbf{g}^{\Am}_{V({b})}) \big )
\end{eqnarray}

where $\Rightarrow$ is used to mean ``localiizes to''. Similarily, 

\begin{eqnarray}
\big( \prod_{b \in {\B}} P(r_b \mid s_b) \prod_{k=1}^{n(b)} P(g_{b,k} \mid r_b) \big) &\Rightarrow& \big( \prod_{a \in \Am} P(r_a \mid s_a) \prod_{k=1}^{n(a)} P^{\Am}(g_{a,k} \mid r_a) \big) \\
%
P^{\Am}(g_{a,k} \mid r_a) &=& \begin{cases} \label{actPg} 1-\sum_{a' \in {\Am}} P(g_{a,k} = a' \mid r_a) \mbox{ if } g_{a,k} = \emptyset \\
 P(g_{a,k} \mid r_a) \mbox{ , otherwise } \end{cases}
\end{eqnarray}

\begin{eqnarray}
\big( \prod_{b \in {\B}} P(\qb \mid s_b)\big) \Rightarrow \big( \prod_{a \in {\Am}} P(\qa \mid s_a)\big)
\end{eqnarray}
\begin{eqnarray}
P(Y \mid \mathbf{q}) \Rightarrow P^{\Am}(Y \mid \mathbf{q^{\Am}})
\end{eqnarray}

where we evaluate $P^{\Am}(Y \mid \mathbf{q^{\Am}})$ in a similar fashion to $P(Y \mid \mathbf{q})$ outlined in Eqn. \ref{eqn.Yq}. We note that $P^{\Am}(Y \mid \mathbf{q^{\Am}}) \neq P(Y \mid \mathbf{q^{\Am}})$ since we do not perform the correct marginalizations.

We define the overall localized distribution, $P^{\Am}(X^{\Am}$, Y), in terms of the previously defined localization distributions:
\begin{eqnarray}
P^{\Am}(X^{\Am}, Y)
&=& \big(\prod_{a \in {\Am}} P^{\Am}(s_a \mid \mathbf{g}^{\Am}_{V({b})}) \big ) \label{eqn.localJoint}\\
& & \times \big( \prod_{a \in {\Am}} P(r_a \mid s_a) \prod_{k=1}^{n(a)} P^{\Am}(g_{a,k} \mid r_a) \big) \nonumber \\
& & \times \prod_{a \in {\Am}} P(\qa \mid s_a)\big) \nonumber \\
& & \times \big( P^{\Am}(Y \mid \mathbf{q^{\Am}}) \big) \nonumber
\end{eqnarray}

\section*{State representation}

Representing the configuration of an image for a \textbf{non-localized} model requires specifying $s_b$, $g_b$, $r_b$, $\qb$, $\forall b \in \B$. For a \textbf{localized} model, the same random variables could be used as the state representation (but with $\mathbf{g^{\Am}}$ defined over a different domain than $\mathbf{g}$). However, requiring specification of the $r_b$'s is unnecessary and too restrictive; we may not be able to well predict the rule a brick has used by examining the set of active bricks. This is especially true if the active set only contains one brick! Instead, we represent the state representation of the localized model by the random variables $\Xrm = \{s_a$, $g_a$, $\qa\}$, $\forall a \in \Am$, and we marginalize over $r_a$ wherever it is used. In particular

\begin{eqnarray}
P^{\Am}(\Xrm, Y) &=& \sum_{\mathbf{r}} P^{\Am}(\X^{\Am}, Y)  \\
&=& \big( \prod_{a \in {\Am}} P^{\Am}(s_a \mid \mathbf{g}^{\Am}_{V({b})}) \big ) \label{eqn.localJointXrm} \\
& & \times \big( \prod_{a \in {\Am}} \sum_{r_a} \big(P(r_a \mid s_a) {\displaystyle \prod_{k=1}^{n(a)}} P^{\Am}(g_{a,k} \mid r_a) \big) \big) \nonumber \\
& & \times \prod_{a \in {\Am}} P(\qa \mid s_a) \nonumber \\
& & \times \big( P^{\Am}(Y \mid \mathbf{q^{\Am}}) \big). \nonumber
\end{eqnarray}

\section*{Inference}

Inference proceeds by iteratively adding one brick at a time to the active set. At any iteration, $m$, our goal is to maintain a representation of the localized posterior distribution $P^{\Am}(\Xrm \mid Y)$. To maintain this representation, we employ a particle filtering approach. We treat samples from the previous localized posterior distribution $P^{\A_{m-1}}(\X^{\A_{m-1}}_{-r} \mid Y)$ as draws from a proposal distribution, and re-weight these particles accordingly (discuss re-weighting and evaluating marginals later). Note that $P^{\A_{m-1}}(\X^{\A_{m-1}}_{-r} \mid Y) \neq P^{\A_{m}}(\X^{\A_{m-1}}_{-r} \mid Y)$; the model defined by $P^{\A_{m}}$ is different than the one defined by $P^{\A_{m-1}}$.

Suppose we have sampled a previous state $\X^{\A_{m-1}}_{-r}$, and suppose the brick we are adding to the active set is denoted $b$. To generate $\Xrm$, we need to draw values for the set of random variables defined by $\Xrm \setminus \X^{\A_{m-1}}_{-r} = \{s_b$, $\qb$, $\mathbf{g^{\Am}_b}$, $comp(g^{\A_{m-1}})\}$. That is, we need to ``expand'' the representation of $\X^{\A_{m-1}}_{-r}$ to include brick $b$. As such, we need to draw a sample from $P^{\Am}(s_b, \qb, \mathbf{g^{\Am}_b}, comp(g^{\A_{m-1}}) \mid \X^{\A_{m-1}}_{-r}, Y)$. We note that for $g \in \gcomp$, the only possible values that can be sampled for $g$ are $g = \emptyset$, $g= b$. That is, each not-yet-assigned brick slot can choose either to point to the newly-added active brick $b$, or remain empty. Note that
\begin{eqnarray}
	P^{\Am}(s_b, \qb, \mathbf{g^{\Am}_b}, comp(g^{\A_{m-1}}) \mid \X^{\A_{m-1}}_{-r}, Y) &=& \frac{P^{\Am}(s_b, \qb, \mathbf{g^{\Am}_b}, comp(g^{\A_{m-1}}), \X^{\A_{m-1}}_{-r} \mid Y)}{\sum_{s_b} \sum_{g_b}\sum_{\mathbf{g^{\Am}_b}}\sum_{comp(g^{\A_{m-1}})} P^{\Am}(s_b, \qb, \mathbf{g^{\Am}_b}, comp(g^{\A_{m-1}}), \X^{\A_{m-1}}_{-r} \mid Y)} \label{eqn.localPost1} \\
P^{\Am}(s_b, \qb, \mathbf{g^{\Am}_b}, comp(g^{\A_{m-1}}), \X^{\A_{m-1}}_{-r} \mid Y) &=& P^{\Am}(\Xrm \mid Y) \\
&\propto& P^{\Am}(\Xrm, Y). \label{eqn.localPost}
\end{eqnarray}

We first note that for any setting of the random variables outlined above, we may classify it as one of three events: 1) brick $b$ is off, 2) brick $b$ is on, and has no parents, and 3) brick $b$ is on, and has at least one parent. We compute the mass associated with each of these three events, with approximations where needed. We will be able to evaluate the denominator in Eqn. \ref{eqn.localPost1} by summing the probability mass associated with these three events. We examine these events in more detail below.

\subsection*{Event 1: brick $b$ is off}

For this event, there is only one setting of the random variables $\{s_b, \qb, \vec{g}^{\Am}_b, comp(g^{\A_{m-1}}\}$ that has non-zero probability: $s_b = 0$, $\qb = \perp$, $g_{b,k} = \perp \forall k$, $g = \emptyset$ $\forall g \in \gcompPrev$. Note that $g^{\A_{m}} \neq g^{\A_{m-1}}$ only because $g^{\A_{m-1}}$ specifies the (lack of) children for the brick $b$, which is not specified in $g^{\A_{m-1}}$. The other corresponding entries of $g^{\A_{m}}$ and $g^{\A_{m-1}}$ are indeed equal, however. It is trivial to evaluate Eqn. \ref{eqn.localJointXrm} given a particular sampled previous state, $\X^{\A_{m-1}}_{-r}$.

\subsection*{Event 2: brick $b$ is on and has no parents}

There are many settings of the random variables $\{s_b, \qb, \mathbf{g^{\Am}_b}, comp(g^{\A_{m-1}})\}$ that fall under this event. They can be characterized as: $s_b=1$, $\qb \neq \perp$ (or equivalently, $\qb \in P(b)$), $g = \emptyset$ $\forall g \in \gcompPrev$, and $g^{\Am}_{b,k} = \{\emptyset, a \text{ s.t. } a \in {\Am}\}$, $\forall k$. So, the probability mass associated with this event is given by
\begin{eqnarray}
\text{P(Event 2)} = \sum_{\vec{q}_b \in P(b)} \sum_{\vec{g}^{\Am}_{b}} P^{\Am}(s_b=1, \vec{q}_b, \vec{g}^{\Am}_{b}, \gcompPrev = \emptyset, \XrmPrev, Y) \label{eqn.event2}
\end{eqnarray}

where we have used $\gcompPrev = \emptyset$ as a shorthand for $g = \emptyset$ $\forall g \in \gcompPrev$. Expanding, we get

\begin{eqnarray}
% \sum_{\vec{q}_b \neq \perp} \sum_{\vec{g}^{\Am}_{b}} P^{\Am}(s_b=1, \vec{q}_b, \vec{g}^{\Am}_{b} \gcompPrev = \emptyset, \X^{\A_{m-1}}, I)  \nonumber \\
&=& \sum_{\vec{g}^{\Am}_{b}} \big( \prod_{a \in {\Am}} P^{\Am}(s_a \mid \mathbf{g}^{\Am}_{V({b})}) \big ) \big( \prod_{a \in {\A_{m}}} \big( \sum_{r_a} P(r_a \mid s_a) \prod_{k=1}^{n(a)} P^{\Am}(g_{a,k} \mid r_a) \big) \big) \label{eqn.child} \\
%
& & \big( \prod_{a \in {\A_{m-1}}} P(\qa \mid s_a)\big) \big( \sum_{\vec{q}_b \in P(b)} P(\qb \mid s_b = 1) P^{\Am}(Y \mid \mathbf{q^{\A_{m-1}}}, \vec{q}_b) \big). \nonumber
\end{eqnarray}

The second term in Eqn. \ref{eqn.child}, $\big( \prod_{a \in {\A_{m-1}}} P(\qa \mid s_a)\big) \big( \sum_{\vec{q}_b \in P(b)} P(\qb \mid s_b = 1)\big) P^{\Am}(Y \mid \mathbf{q^{\A_{m-1}}}, \vec{q}_b) \big)$, can be evaluated by enumerating all possible values for $\vec{q}_b \in P(b)$. In practice, whether this is practical or not is dependent on $\#(P(b))$. If $\#(P(b))$ is ``small enough'', then $\big( \prod_{a \in {\A_{m-1}}} P(\qa \mid s_a)\big) \big( \sum_{\vec{q}_b \neq \perp} P(\qb \mid s_b = 1)\big) P^{\Am}(Y \mid \mathbf{q^{\A_{m-1}}}, \vec{q}_b) \big)$ can be computed. 

Turning our attention to the first term in Eqn. \ref{eqn.child}, we note that it is not possible to efficiently compute this quantity. So we come up with approximations for it.

From the definition given in Eqn. \ref{eqn.probS}, 
%
\begin{eqnarray}
\prod_{a \in {\Am}} P^{\Am}(s_a \mid \mathbf{g}^{\Am}_{V({b})}) = \epsilon^{\#(orph(\Xrm))} \times (1-\epsilon)^{\#(off(\A_{m-1}))}. \label{eqn.orphan}
\end{eqnarray}
%
So to evaluate $\prod_{a \in {\Am}} P^{\Am}(s_a \mid \mathbf{g}^{\Am}_{V({b})})$ it is only necessary to find the number of orphan bricks in $\Am$.
%
Next, we derive a relationship between $\#(orph(\XrmPrev))$ and $\#(orph(\Xrm))$. If $s_b=1$ and brick $b$ does not have at least one parent, then the number of orphans increases by $1$ since $b$ itself is an orphan. If we denote the \textbf{set} of children of $b$ given a particular $\vec{g}_b^{\Am}$ as $ch(\vec{g}_b^{\Am})$, then the number of orphans is decreased by $\#(ch(\vec{g}_b^{\Am}) \cap (orph(\XrmPrev))$ due to $b$'s becoming the parent of these previously-orphaned bricks, and $b$ being unable to be its own child. The relation between $\#(orph(\Xrm))$ and $\#(orph(\XrmPrev))$ can thus be summarized as
%
\begin{eqnarray}
\#(orph(\Xrm)) &=& \#(orph(\XrmPrev)) + I(\text{b is orphan}) - \#(ch(\vec{g}_b^{\Am}) \cap orph(\XrmPrev)) \\
%
&=& \#(orph(\XrmPrev)) + I(\text{b is orphan}) - \#(H^{\X_{m-1}}(\vec{g}_b^{\Am})) \label{eqn.orphanM1}
\end{eqnarray}
%
Combining Eqn. \ref{eqn.orphanM1} with Eqn. \ref{eqn.orphan} yields
%
\begin{eqnarray}
\prod_{a \in {\Am}} P^{\Am}(s_a \mid \mathbf{g}^{\Am}_{V({b})}) &=& \epsilon^{\#(orph(\Xrm))} \times (1-\epsilon)^{\#(off(\A_{m-1}))} \\
%
&=&\epsilon^{\#(orph(\XrmPrev)) + I(\text{b is orphan}) - \#(H^{\X_{m-1}}(\vec{g}_b^{\Am}))} \times (1-\epsilon)^{\#(s(\A_{m-1}))} \\
%
&=& \epsilon^{\#(orph(\XrmPrev)) + I(\text{b is orphan})} \times \epsilon^{- \#(H^{\X_{m-1}}(\vec{g}_b^{\Am}))} \times (1-\epsilon)^{\#(off(\A_{m-1}))} \label{eqn.orphan2}.
\end{eqnarray}
%
Combining Eqn. \ref{eqn.orphan2} and the first term of Eqn. \ref{eqn.child}, and noting for this event that $ I(\text{b is orphan})=1$ yields
\begin{eqnarray}
& &\sum_{\vec{g}^{\Am}_{b}} \big( \prod_{a \in {\Am}} P^{\Am}(s_a \mid \mathbf{g}^{\Am}_{V({b})})  \big( \prod_{a \in {\A_{m}}} \sum_{r_a} P(r_a \mid s_a) \prod_{k=1}^{n(a)} P^{\Am}(g_{a,k} \mid r_a) \big) \big ) \\
%
&=& \sum_{\vec{g}^{\Am}_{b}} \big(\epsilon^{\#(orph(\XrmPrev)) + I(\text{b is orphan})} \times (1-\epsilon)^{\#(off(\A_{m-1}))} \times \epsilon^{- \#(H^{\X_{m-1}}(\vec{g}_b^{\Am}))}  \nonumber \\
& & \times \big( \prod_{a \in {\A_{m}}} \sum_{r_a} P(r_a \mid s_a) \prod_{k=1}^{n(a)} P^{\Am}(g_{a,k} \mid r_a) \big) \big) \\
%
&=& \epsilon^{\#(orph(\XrmPrev)) + 1} \times (1-\epsilon)^{\#(off(\A_{m-1}))} \nonumber \\
& & \times  \sum_{\vec{g}^{\Am}_{b}} \big( \epsilon^{- \#(H^{\X_{m-1}}(\vec{g}_b^{\Am}))}  \big( \prod_{a \in {\A_{m}}} \sum_{r_a} P(r_a \mid s_a) \prod_{k=1}^{n(a)} P^{\Am}(g_{a,k} \mid r_a) \big) \big) \\
%
&=& \epsilon^{\#(orph(\XrmPrev)) + 1} \times (1-\epsilon)^{\#(off(\A_{m-1}))} \nonumber \\
& &\times  \big( \prod_{a \in {\A_{m}}, a \neq b} \sum_{r_a} P(r_a \mid s_a) \prod_{k=1}^{n(a)} P^{\Am}(g_{a,k} \mid r_a) \big) \\
& & \times \sum_{\vec{g}^{\Am}_{b}} \big( \epsilon^{- \#(H^{\X_{m-1}}(\vec{g}_b^{\Am}))} \big( \sum_{r_b} P(r_b \mid s_b = 1) \prod_{k=1}^{n(b)} P^{\Am}(g_{b,k} \mid r_b) \big) \big) \nonumber \\
%
&=& \epsilon^{\#(orph(\XrmPrev)) + 1} \times (1-\epsilon)^{\#(off(\A_{m-1}))} \label{bottomup}  \\
& & \times \big( \prod_{a \in {\A_{m}}, a \neq b} \sum_{r_a} P(r_a \mid s_a) \prod_{k=1}^{n(a)} P^{\Am}(g_{a,k} \mid r_a) \big)  \\
& & \times \sum_{r_b} \big(P(r_b \mid s_b = 1) \sum_{\vec{g}^{\Am}_{b}} \big( \epsilon^{-\#(H^{\X_{m-1}})}(\vec{g}^{\Am}_b) \prod_{k=1}^{n(b)} P^{\Am}(g_{b,k} \mid r_b) \big) \big) \nonumber
\end{eqnarray} 

It is intractable to compute the quantity given in Eqn. \ref{bottomup}. Note that the evaluation of $\epsilon^{-\#(H^{\X_{m-1}}(\vec{g}^{\Am}_b))}$ for a given $\vec{g}^{\Am}_{b}$ does not factorize over the $g_{b,k}$'s. Recall that $\#(H^{\X_{m-1}}(\vec{g}^{\Am}_b))$ is the number of \emph{distinct} children pointed to $\vec{g}^{\Am}_b$ that were previously orphans according to $\X_{m-1}$; it is the requirement that children be distinct that couples the $g^{\Am}_{b,k}$'s. If we remove the distinctness requirement, and simply count the number of previously-orphan children pointed to by $\vec{g}^{\Am}_b$, then the effect of each of the $g^{\Am}_{b,k}$'s for a given $r_b$ will decouple. More precisely, we use the approximation

\begin{eqnarray}
\#(H^{\X_{m-1}}(\vec{g}^{\Am}_b)) &\approx& \#(\tilde{H}^{\X_{m-1}} (\vec{g}^{\Am}_b)) \\
\implies \epsilon^{-\#(H^{\X_{m-1}}(\vec{g}^{\Am}_b))} &\approx& \epsilon^{-\#(\tilde{H}^{\X_{m-1}}(\vec{g}^{\Am}_b))} \label{approx1}.
\end{eqnarray}

(OK, the above isn't really true. Just because the quantities in the exponents are roughly equally doesn't necessarily mean the entire exponentiated quantity is.).
 
We expect the approximation given in Eqn. \ref{approx1} to be a good approximation when the probability of the event $\exists \{k,k'\}, k \neq k'$ s.t. $g^{\Am}_{b,k}=g^{\Am}_{b,k'}$ is small. That is, it is unlikely that two slots will point to the same child.

With the approximation outlined in Eqn. \ref{approx1}, we can approximately compute the second term in $\ref{bottomup}$. For a given $r_b$, we can compute the probability that each of the $n(b)$ slots will point to a previously-orphaned child separately. We then average this quantity over the $r_b$'s. In other words, with the approximation in Eqn. \ref{approx1}, the computation factorizes across the slots for a given $r_b$:

\begin{eqnarray}
& & \sum_{r_b} \big(P(r_b \mid s_b = 1) \sum_{\vec{g}^{\Am}_{b}} \big( \epsilon^{-\#(H^{\X_{m-1}})}(\vec{g}^{\Am}_b) \prod_{k=1}^{n(b)} P^{\Am}(g_{b,k} \mid r_b) \big) \big) \\
%
&\approx& \sum_{r_b} \big(P(r_b \mid s_b = 1) \sum_{\vec{g}^{\Am}_{b}} \big( \epsilon^{-\#(\tilde{H}^{\X_{m-1}})}(\vec{g}^{\Am}_b) \prod_{k=1}^{n(b)} P^{\Am}(g_{b,k} \mid r_b) \big) \big) \label{eqn.bottomupTilde}  \\
%
&=& \sum_{r_b} \big( P(r_b \mid s_b = 1) \sum_{\vec{g}^{\Am}_{b}} \big( \prod_{k=1}^{n(b)} \epsilon^{-\#(\tilde{H}^{\X_{m-1}}(g^{\Am}_{b,k}))}P^{\Am}(g_{b,k} \mid r_b) \big) \big) \\
%
&=& \sum_{r_b} \big( P(r_b \mid s_b = 1) \prod_{k=1}^{n(b)} \big( \sum_{g^{\Am}_{b,k}} \epsilon^{-\#(\tilde{H}^{\X_{m-1}}(g^{\Am}_{b,k}))}P^{\Am}(g_{b,k} \mid r_b) \big) \big) \\
%
&=& \sum_{r_b} \big( P(r_b \mid s_b = 1) \prod_{k=1}^{n(b)} \big ( \epsilon^{-1}P^{\Am}(g_{b,k} = a, a \in orph(\XrmPrev) \mid r_b) \\ 
& & \qquad \qquad \qquad \qquad \qquad + \epsilon^{0}(1-P^{\Am}(g_{b,k} = a, a \in orph(\XrmPrev) \mid r_b)) \big) \big). \nonumber \label{eqn.approxBottomUpFinal}
\end{eqnarray}

Notice the $~\tilde{}$ above the $H$ in Eqn. \ref{eqn.bottomupTilde}. It is trivial to compute $P^{\Am}(g_{b,k} = a, a \in orph(\XrmPrev) \mid r_b)$ for each rule $r_b$ since $P^{\Am}(g_{b,k} \mid r_b)$ is a \emph{Discrete} distribution (table of probabilities) and the set $orph(\XrmPrev)$ can be easily enumerated.

Combining the derived approximation, Eqn. \ref{eqn.approxBottomUpFinal}, Eqn. \ref{bottomup}, and Eqn. \ref{eqn.child} yields
%
\begin{eqnarray}
P(\text{Event 2}) &\approx& \epsilon^{\#(orph(\XrmPrev)) + 1} \times (1-\epsilon)^{\#(off(\A_{m-1}))}  \label{eqn.even2final} \\
%
&\times& \prod_{a \in {\A_{m-1}}} \big( \sum_{r_a} P(r_a \mid s_a) \prod_{k=1}^{n(a)} P^{\Am}(g_{a,k} \mid r_a) \big) \nonumber  \\
%
&\times&  \big( \sum_{r_b} \big( P(r_b \mid s_b = 1) \prod_{k=1}^{n(b)} \big ( \epsilon^{-1}P^{\Am}(g_{b,k} = a, a \in orph(\XrmPrev) \mid r_b) \nonumber \\ 
%
& & \qquad \qquad \qquad \qquad \qquad + \epsilon^{0}(1-P^{\Am}(g_{b,k} = a, a \in orph(\XrmPrev) \mid r_b)) \big) \big) \big) \nonumber \\
%
&\times& \big( \prod_{a \in {\A_{m-1}}} P(\qa \mid s_a) \big) \nonumber \\
&\times& \sum_{\vec{q}_b \in P(b)} \big( P(\qb \mid s_b = 1) P^{\Am}(Y \mid \mathbf{q^{\A_{m-1}}}, \vec{q}_b) \big) \big) \nonumber 
\end{eqnarray} 

where we have made use of the equality $\{ a: a \in \Am, a \neq b\} = \{a : a \in \A_{m-1}$ \}.

Now do alternate approximation (Jeroen's approx)...although Jeroen's approx is probably worse.

\subsection*{Event 3: brick $b$ is on, and has at least $1$ parent}

There are many settings of the random variables $\{s_b, \qb, \mathbf{g^{\Am}_b}, comp(g^{\A_{m-1}})\}$ that fall under this event. They can be characterized as: $s_b=1$, $\qb \neq \perp$ (or equivalently, $\qb \in P(b)$), $\{\gcompPrev : \exists \{a,k\}$ s.t. $ g^{\Am}_{a,k} = b \}$, and $g^{\Am}_{b,k} = \{\emptyset, a \text{ s.t. } a \in {\Am}\}$, $\forall k$. For brevity, we will use the shorthand $\gcompPrev = b$ for $\{\gcompPrev : \exists \{a,k\}$ s.t. $ g^{\Am}_{a,k} = b \}$

So, the probability of this event is given by
\begin{eqnarray}
P(\text{Event 3}) &=& \sum_{\gcompPrev = b} \sum_{\vec{q}_b \in P(b)} \sum_{\vec{g}^{\Am}_{b}} P^{\Am}(s_b=1, \vec{q}_b, \vec{g}^{\Am}_{b}, \gcompPrev, \XrmPrev, Y) \\
%
&=& \sum_{\gcompPrev = b} \sum_{\vec{g}^{\Am}_{b}} \big( \prod_{a \in {\Am}} P^{\Am}(s_a \mid \mathbf{g}^{\Am}_{V({b})}) \big ) \big( \prod_{a \in {\A_{m}}} \big( \sum_{r_a} P(r_a \mid s_a) \prod_{k=1}^{n(a)} P^{\Am}(g_{a,k} \mid r_a) \big) \big) \label{eqn.topdown2} \\
%
& & \big( \prod_{a \in {\A_{m-1}}} P(\qa \mid s_a)\big) \big( \sum_{\vec{q}_b \in P(b)} P(\qb \mid s_b = 1) P^{\Am}(Y \mid \mathbf{q^{\A_{m-1}}}, \vec{q}_b) \big) \nonumber
\end{eqnarray}

where we have already outlined how to deal with $\big( \prod_{a \in {\A_{m-1}}} P(\qa \mid s_a)\big) \big( \sum_{\vec{q}_b \in P(b)} P(\qb \mid s_b = 1) P^{\Am}(Y \mid \mathbf{q^{\A_{m-1}}}, \vec{q}_b) \big)$ in detailing how to approximate the probability of Event 2. So, we focus on computing $\sum_{\gcompPrev}  \sum_{\vec{g}^{\Am}_{b}} \big( \prod_{a \in {\Am}} P^{\Am}(s_a \mid \mathbf{g}^{\Am}_{V({b})}) \big ) \big( \prod_{a \in {\A_{m}}} \big( \sum_{r_a} P(r_a \mid s_a) \prod_{k=1}^{n(a)} P^{\Am}(g_{a,k} \mid r_a) \big) \big)$.

We present two ways to compute Eqn. \ref{eqn.topdown2} efficiently.


\subsubsection*{Jeroen's method: 1- trick}

Let us rewrite the first term of Eqn. \ref{eqn.topdown2}.

\begin{eqnarray}
& & \sum_{\gcompPrev = b} \sum_{\vec{g}^{\Am}_{b}} \big( \prod_{a \in {\Am}} P^{\Am}(s_a \mid \mathbf{g}^{\Am}_{V({a})}) \big ) \big( \prod_{a \in {\A_{m}}} \big( \sum_{r_a} P(r_a \mid s_a) \prod_{k=1}^{n(a)} P^{\Am}(g_{a,k} \mid r_a) \big) \big)  \\
&=& \epsilon^{\#(orph(\XrmPrev))} \times (1-\epsilon)^{\#(off(\A_{m-1}))}  \label{eqn.topdownAlternate} \\ 
& & \times \sum_{\gcompPrev = b} P^{\Am}(s_{a_m} \mid \mathbf{g}^{\Am}_{V({a_m})}) \big( \prod_{t=1}^{m-1} \sum_{r_{a_t}} P(r_{a_t} \mid s_{a_t}) \prod_{k=1}^{n({a_t})} P^{\Am}(g_{{a_t},k} \mid r_{a_t}) \big)  \nonumber \\
& & \times \sum_{r_{a_m}} \big(P(r_{a_m} \mid s_{a_m} = 1) \sum_{\vec{g}^{\Am}_{{a_m}}} \big( \epsilon^{-\#(H^{\X_{m-1}})}(\vec{g}^{\Am}_{a_m}) \prod_{k=1}^{n({a_m})} P^{\Am}(g_{{a_m},k} \mid r_{a_m}) \big) \big) \big). 
\end{eqnarray}

The middle line of Eqn. \ref{eqn.topdownAlternate} requires further analysis- the computation of the other lines in the equation of been previously analyzed, and an efficient method of (approximate) computation outlined. We can rewrite the middle line of Eqn. \ref{eqn.topdownAlternate} as:
%
\begin{eqnarray}
&=& \sum_{\gcompPrev=b} \big( P^{\Am}(s_{a_m} \mid \mathbf{g}^{\Am}_{V({a_m})}) \\
& & \times \prod_{t=1}^{m-1} \big( \sum_{r_{a_t}} P(r_{a_t} \mid s_{a_t}) \big( \prod_{\substack{k : \vec{g}^{\A{m-1}}_{a_t,k} \not \in \\ \gcompPrev }} P^{\Am}(g_{{a_t},k} \mid r_{a_t}) \big) \big( \prod_{\substack{k : \vec{g}^{\A{m-1}}_{a_t,k} \in \\ \gcompPrev }} P^{\Am}(g_{{a_t},k} = \mid r_{a_t}) \big) \big) \big) \nonumber\\
%
&=& \big( \sum_{\gcompPrev} \big( P^{\Am}(s_{a_m} \mid \mathbf{g}^{\Am}_{V({a_m})}) \label{eqn.temp3} \\
& & \times \prod_{t=1}^{m-1} \big( \sum_{r_{a_t}} P(r_{a_t} \mid s_{a_t}) \big( \prod_{\substack{k : \vec{g}^{\A{m-1}}_{a_t,k} \not \in \\ \gcompPrev }} P^{\Am}(g_{{a_t},k} \mid r_{a_t}) \big) \big( \prod_{\substack{k : \vec{g}^{\A{m-1}}_{a_t,k} \in \\ \gcompPrev }} P^{\Am}(g_{{a_t},k} \mid r_{a_t}) \big) \big) \big) \big) \nonumber\\
& &- \big( \epsilon \times \prod_{t=1}^{m-1} \big( \sum_{r_{a_t}} P(r_{a_t} \mid s_{a_t}) \big( \prod_{\substack{k : \vec{g}^{\A{m-1}}_{a_t,k} \not \in \\ \gcompPrev }} P^{\Am}(g_{{a_t},k} \mid r_{a_t}) \big) \big( \prod_{\substack{k : \vec{g}^{\A{m-1}}_{a_t,k} \in \\ \gcompPrev }} P^{\Am}(g_{{a_t},k} = \emptyset \mid r_{a_t}) \big) \big) \big) \big) \nonumber
\end{eqnarray}

where we have used the fact that $\sum_{\gcompPrev=b}(\cdot) = \sum_{\gcompPrev}(\cdot) - \sum_{\gcompPrev= \emptyset} (\cdot)$ since $\{ g=\{\emptyset, b\}, \forall g \in \gcompPrev\}$. We now observe that for all settings of $\gcompPrev$ except $\gcompPrev= \emptyset$, we have $P^{\Am}(s_{a_m} \mid \mathbf{g}^{\Am}_{V({a_m})})$ = 1. That is, for all but one setting of $\gcompPrev$ we can treat $P^{\Am}(s_{a_m} \mid \mathbf{g}^{\Am}_{V({a_m})})$ as a constant, and so we can ``push'' $\gcompPrev$ through $P^{\Am}(s_{a_m} \mid \mathbf{g}^{\Am}_{V({a_m})})$. We compensate for the fact that we use the wrong value of $P^{\Am}(s_{a_m} \mid \mathbf{g}^{\Am}_{V({a_m})})$ for the single case of $\gcompPrev= \emptyset$ by adding an appropriate correction factor. So, we can write  Eqn. \ref{eqn.temp3} as

\begin{eqnarray}
&=&  \prod_{t=1}^{m-1} \big( \sum_{r_{a_t}} P(r_{a_t} \mid s_{a_t}) \big( \prod_{\substack{k : \vec{g}^{\A{m-1}}_{a_t,k} \not \in \\ \gcompPrev }} P^{\Am}(g_{{a_t},k} \mid r_{a_t}) \big)\big) \\
%
& & + \underbrace{(\epsilon-1) \big( \prod_{t=1}^{m-1} \big( \sum_{r_{a_t}} P(r_{a_t} \mid s_{a_t}) \big( \prod_{\substack{k : \vec{g}^{\A{m-1}}_{a_t,k} \not \in \\ \gcompPrev }} P^{\Am}(g_{{a_t},k} \mid r_{a_t}) \big) \big( \prod_{\substack{k : \vec{g}^{\A{m-1}}_{a_t,k} \in \\ \gcompPrev }} P^{\Am}(g_{{a_t},k} =\emptyset \mid r_{a_t}) \big)\big)}_{\text{correction factor}} \nonumber \\
%
& &- \big(\epsilon \times \prod_{t=1}^{m-1} \big( \sum_{r_{a_t}} P(r_{a_t} \mid s_{a_t}) \big( \prod_{\substack{k : \vec{g}^{\A{m-1}}_{a_t,k} \in \\ \gcompPrev }} P^{\Am}(g_{{a_t},k} \mid r_{a_t}) \big) \big( \prod_{\substack{k : \vec{g}^{\A{m-1}}_{a_t,k} \not \in \\ \gcompPrev }} P^{\Am}(g_{{a_t},k} = \emptyset \mid r_{a_t}) \big) \big) \big) \big) \nonumber \\
%
&=&  \prod_{t=1}^{m-1} \big( \sum_{r_{a_t}} P(r_{a_t} \mid s_{a_t}) \big( \prod_{\substack{k : \vec{g}^{\A{m-1}}_{a_t,k} \not \in \\ \gcompPrev }} P^{\Am}(g_{{a_t},k} \mid r_{a_t}) \big)\big) \\
%
& & -\big( \prod_{t=1}^{m-1} \big( \sum_{r_{a_t}} P(r_{a_t} \mid s_{a_t}) \big( \prod_{\substack{k : \vec{g}^{\A{m-1}}_{a_t,k} \not \in \\ \gcompPrev }} P^{\Am}(g_{{a_t},k} \mid r_{a_t}) \big) \big( \prod_{\substack{k : \vec{g}^{\A{m-1}}_{a_t,k} \in \\ \gcompPrev }} P^{\Am}(g_{{a_t},k} =\emptyset \mid r_{a_t}) \big)\big). \nonumber
\end{eqnarray}

We can now compute Eqn. \ref{eqn.topdown2} using previously defined approximations/computations.

\subsubsection*{Pedro's method: string of $0$'s and $1$'s}

Recall that $g = \{b, \emptyset\}, \forall g \in \gcompPrev$ for this event, and note that the $g \in \gcompPrev$'s are \textbf{not} conditionally independent given $s_b$ (see the graphical model in Fig. \ref{gm}) due to explaining away; the term that ``ties'' them together is  $P(s_b = 1 \mid \mathbf{g}_{V(b)})$. However, we use the particular definition of  $P(s_b = 1 \mid \mathbf{g}_{V(b)})$  given in Eqn. \ref{eqn.probS} to derive an efficient way of computing the first term in Eqn. \ref{eqn.topdown2} exactly. In specific, we use the fact that  the quantity $P(s_b = 1 \mid \mathbf{g}_{V(b)})$  depends only on whether there exists at least one brick that has $b$ as a child.

Suppose for some $\{k,m'\}, g_{m',k} = b, g_{m',k} \in \gcompPrev$. Then, $P(s_b = 1 \mid \mathbf{g}_{V(b)}) = 1$ regardless of the values the other random variables in $\gcompPrev$ take. So, the remaining random variables $\gcompPrev$ become conditionally independent given that $\exists \{k,m'\}, g_{m',k} = b, g_{m',k} \in \gcompPrev$.

To take advantage of the above observation, we partition the event $\gcompPrev = b$ into $m-1$ distinct events. Recall that the bricks in
$\A_{m-1}$ are ordered, and each brick is in $\A_{m-1}$  can be denoted as $a_{m'}, m' \in [1...m-1]$. We define the $m$th event as the event that brick $a_{m}$ takes brick $b$ as a child, and brick $a_{m'}$ does not take $b$ as a child $\forall m' < m$. Formally, the $m$th event, which we denote as $e_m$, is is defined as $e_m = \{\gcompPrev : \exists k$ s.t. $g^{\Am}_{m,k} = b, \nexists k$ s.t. $g^{\Am}_{m',k} = b, \forall m' < m \}$. Another way to understand the partitioning of event space is to imagine string of length $m-1$, consisting of $0$'s and $1$'s, where an entry $m'$ is $1$ if brick $m'$ is a parent of brick $b$, and $0$ otherwise. Then, the $m$th event is represented as the set of strings with the prefix $\{ \{0\}^{m-1},1\}$. With this representation, it should be clear that each $m$ event is mutually exclusive since no two strings can have the same prefix, and the set of $M$ events covers all events where brick $b$ has at least one parent (\emph{ie} the string representation contains at least one $1$). Therefore, the set of events we have defined is indeed a partitioning of the event that brick $b$ has at least one parent.

We can then write the first term in Eqn. \ref{eqn.topdown2} as

\begin{eqnarray}
&=&\sum_{\gcompPrev = b} \sum_{\vec{g}^{\Am}_{b}} \big( \prod_{a \in {\Am}} P^{\Am}(s_a \mid \mathbf{g}^{\Am}_{V({b})}) \big ) \big( \prod_{a \in {\A_{m}}} \big( \sum_{r_a} P(r_a \mid s_a) \prod_{k=1}^{n(a)} P^{\Am}(g_{a,k} \mid r_a) \big) \big) \\
%
&=& \sum_{n=1}^{m-1} \sum_{\gcompPrev \in e_{n}} \big(  \sum_{\vec{g}^{\Am}_{b}} \big( \prod_{a \in {\Am}} P^{\Am}(s_a \mid \mathbf{g}^{\Am}_{V({b})}) \big ) \big( \prod_{a \in {\A_{m}}} \big( \sum_{r_a} P(r_a \mid s_a) \prod_{k=1}^{n(a)} P^{\Am}(g_{a,k} \mid r_a) \big) \big) \big)  \label{eqn.temp1} \\
%
&=& \sum_{n=1}^{m-1} \sum_{e_{n}} \epsilon^{\#(orph(\XrmPrev))} \times (1-\epsilon)^{\#(off(\A_{m-1}))}  \label{eqn.temp2} \\ 
& & \times \big( \prod_{t=1}^{m-1} \sum_{r_{a_t}} P(r_{a_t} \mid s_{a_t}) \prod_{k=1}^{n({a_t})} P^{\Am}(g_{{a_t},k} \mid r_{a_t}) \big)  \\
& & \times \sum_{r_{a_m}} \big(P(r_{a_m} \mid s_{a_m} = 1) \sum_{\vec{g}^{\Am}_{{a_m}}} \big( \epsilon^{-\#(H^{\X_{m-1}})}(\vec{g}^{\Am}_{a_m}) \prod_{k=1}^{n({a_m})} P^{\Am}(g_{{a_m},k} \mid r_{a_m}) \big) \big) \big) \nonumber \\
%
&=& \epsilon^{\#(orph(\XrmPrev))} \times (1-\epsilon)^{\#(off(\A_{m-1}))} \\ 
& & \times \sum_{n=1}^{m-1} \sum_{e_{n}} \big( \prod_{t=1}^{m-1} \sum_{r_{a_t}} P(r_{a_t} \mid s_{a_t}) \prod_{k=1}^{n({a_t})} P^{\Am}(g_{{a_t},k} \mid r_{a_t}) \big)  \\
& & \times \sum_{r_{a_m}} \big(P(r_{a_m} \mid s_{a_m} = 1) \sum_{\vec{g}^{\Am}_{{a_m}}} \big( \epsilon^{-\#(H^{\X_{m-1}})}(\vec{g}^{\Am}_{a_m}) \prod_{k=1}^{n({a_m})} P^{\Am}(g_{{a_m},k} \mid r_{a_m}) \big) \big) \big) \nonumber \\
%
&=& \epsilon^{\#(orph(\XrmPrev))} \times (1-\epsilon)^{\#(off(\A_{m-1}))} \label{topdownfinal} \\ 
& & \times \sum_{n=1}^{m-1} \big( \nonumber \\
& & \qquad \prod_{t=1}^{n-1} \big( \sum_{r_{a_t}} P(r_{a_t} \mid s_{a_t}) \big( \prod_{k : \vec{g}^{\Am}_{a_t,k} \neq \emptyset} P^{\Am}(g_{{a_t},k} \mid r_{a_t}) \big) \big( \prod_{k : \vec{g}^{\Am}_{a_t,k} = \emptyset} P^{\Am}(g_{{a_t},k} = \emptyset \mid r_{a_t}) \big) \big) \nonumber \\
& & \qquad \times \big( \sum_{r_{a_n}} P(r_{a_n} \mid s_{a_n}) \big( \prod_{k : \vec{g}^{\Am}_{a_n,k} \neq \emptyset} P^{\Am}(g_{{a_n},k} \mid r_{a_n}) \big) \big( 1 - \prod_{k : \vec{g}^{\Am}_{a_n,k} = \emptyset} P^{\Am}(g_{{a_n},k} = \emptyset \mid r_{a_n}) \big) \big) \nonumber \\
& & \qquad \times \big( \prod_{t=n+1}^{m} \sum_{r_{a_t}} P(r_{a_t} \mid s_{a_t}) \big( \prod_{k : \vec{g}^{\Am}_{a_t,k} \neq \emptyset} P^{\Am}(g_{{a_t},k} \mid r_{a_t}) \big) \big) \nonumber \\
& & \qquad \times \sum_{r_{a_m}} \big(P(r_{a_m} \mid s_{a_m} = 1) \sum_{\vec{g}^{\Am}_{{a_m}}} \big( \epsilon^{-\#(H^{\X_{m-1}})}(\vec{g}^{\Am}_{a_m}) \prod_{k=1}^{n({a_m})} P^{\Am}(g_{{a_m},k} \mid r_{a_m}) \big) \big) \nonumber \\
& & \big) \nonumber
\end{eqnarray}

where we have equivalently expressed $\prod_{a \in \Am, a \neq b}$ as $\prod_{t=1}^m$, referring to brick as $a_t$ instead of $a$. Also, we have used the result from Eqn. \ref{bottomup} in going from Eqn. \ref{eqn.temp1} to Eqn. \ref{eqn.temp2}, except noting that for this event, brick $b$ does not root itself, hence we have $\epsilon^{\#(orph(\XrmPrev))}$ instead of $\epsilon^{\#(orph(\XrmPrev))+1}$. Lastly, we note that the last line of Eqn. \ref{topdownfinal} is the same as Eqn. \ref{bottomup}, and its approximation is given in Eqn. \ref{eqn.approxBottomUpFinal}.

We can compute the terms in the summation $\sum_{n=1}^{m-1}$ using dynamic programming. $\forall m$, we compute the quantities
%
\begin{eqnarray}
& &\sum_{r_{a_m}} P(r_{a_m} \mid s_{a_m}) \big( \prod_{k : \vec{g}^{\Am}_{a_m,k} \neq \emptyset} P^{\Am}(g_{{a_m},k} \mid r_{a_m}) \big) \big( \prod_{k : \vec{g}^{\Am}_{a_m,k} = \emptyset} P^{\Am}(g_{{a_m},k} = \emptyset \mid r_{a_m}) \big) \\
%
& & \sum_{r_{a_m}} P(r_{a_m} \mid s_{a_m}) \big( \prod_{k : \vec{g}^{\Am}_{a_m,k} \neq \emptyset} P^{\Am}(g_{{a_m},k} \mid r_{a_m}) \big) \big( 1 - \prod_{k : \vec{g}^{\Am}_{a_m,k} = \emptyset} P^{\Am}(g_{{a_m},k} = \emptyset \mid r_{a_m}) \big) \\
%
& & \sum_{r_{a_m}} \big(P(r_{a_m} \mid s_{a_m} = 1) \sum_{\vec{g}^{\Am}_{{a_m}}} \big( \epsilon^{-\#(H^{\X_{m-1}})}(\vec{g}^{\Am}_{a_m}) \prod_{k=1}^{n({a_m})} P^{\Am}(g_{{a_m},k} \mid r_{a_m}) \big) \big)
\end{eqnarray}
%
and piece them together as appropriate when evaluating terms in encapsulated in the summation $\sum_{n=1}^{m-1}$. The computation to compute these quantities takes  $O(m)$ time, and there are $O(m-1)$ combinations to consider for these quantities (the $m-1$ different events outlined above), so computation of Eqn. \ref{topdownfinal} is $O(m)$ overall, which we feel is reasonable.
%

 %, where the $m$th event is the event that $\exists k$ s.t. $g^{\Am}_{c,k} = b$

%\begin{figure}[htbp]
%\begin{center}
%\includegraphics[width=\textwidth]{test.pdf}
%\caption{Graphical model}
%\end{center}
%\end{figure}


\end{document}
